\documentclass{article} 
	\usepackage{fontspec,booktabs, xunicode, xltxtra,amsmath,amssymb,caption,colortbl,float,caption,algorithm,algpseudocode}
	\usepackage{graphicx}
	\usepackage{pgfplots}
	\usepackage{xeCJK}
	\setCJKmainfont{MSYH.ttc}
	\author{孔静—2014K8009929022}
	\title{人工智能——寻找数据的规律}

\begin{document}   
	\maketitle
	\tableofcontents
	\section{概\ 述}
		\paragraph{问题:}\ 
		 
		条件属性X,continuous value, 值域为[a,b], 决策属性Y,值域为$\{0,1\}$

		已知一组数据:${(x_i,y_i)}$, 设计一个方法:计算出一个值c,使得:在区间[a,c]和[c,b]上,X与Y的变化规律一致
		\paragraph{要求:}\

		1. 描述事先的方法,可以使用描述性文字,把方法描述清楚
		
		2. 设计一些数据的例子,画出对应的数据直方图,并显示计算出的区间的划分
	\section{分\ 析}
		\paragraph{idea:}\
		
		背景:$similarity=cos(\theta)=\frac{\textbf{A}\cdot\textbf{B}}{||\textbf{A}||||\textbf{B}||}=\frac{\sum_{i=1}^{n} A_{i} B_{i}}{\sqrt{\sum_{i=1}^{n} A_{i}^{2}}\sqrt{\sum_{i=1}^{n}B_{i}^{2}}}$
		
		简单起见,在区间[a,c]和[c,b]上,X与Y的变化规律一致,那么从图像上来看,两段函数要尽可能地想象,由于X是连续递增的,两边相同;所以忽略X,只考虑Y,如果左右Y向量相似度similarity越大,说明变化规律越一致。
	\section{方\ 法}
		\paragraph{step. 1}\
		
		选择一个合适的区间大小,对X进行划分,计算每一段X上Y的平均值,并画出相应的折线图。
		
		\paragraph{step. 2}\
		
		取[a,c]上按顺序取均值如$(Y_1,Y_2,...,Y_i)$, 视为向量A, 在[c,b]上的折线图里等间距取同个数即i个Y值, 视为向量B, 计算similarity(A,B)。同理在[c,b]上取剩下的均值点视为向量C, 在[a,c]等间距取同样个数均值点视为向量D, 计算similarity(C,D)。
		
		两者相加,和最大的,即为相似度最高的,即为我们所寻找的c点。
		
		\paragraph{step. 3}\
		
		可利用二分查找法寻找,先取中点,再去左右部分中点进行比较,若左边大,选择左边继续查找。
		
		\begin{algorithm}
			\caption{Find The C}
			\begin{algorithmic}
				\Procedure {Find The C}{$X$, $Y$}
				\State $\Delta x = Choose(X)$
				\State Drawpicture($X$, $Y$, $\Delta x$)
				\State mid = (left + right) / 2
				\While {left < right}
				\State leftmid = (left + mid) / 2
				\State rightmid = (mid + right) / 2
				\If {similarity(rightmid) > similarity(leftmid)}
				\State left = mid
				\Else
				\State right = mid
				\EndIf
				\EndWhile
				\State \textbf{return} $mid$
				\EndProcedure
			\end{algorithmic}
		\end{algorithm}
	\section{实\ 例}
		\paragraph{数据}
		\begin{center}
			\begin{tabular}{|c||c|c|c|c|c|c|c|c|c|c|}
				\hline
				X & 1.1 & 1.6 & 2.1 & 3.4 & 3.5 & 3.9 & 5.1 & 7 & 8.1\\
				\hline
				Y & 0 & 1 & 1 & 1 & 0 & 1 & 0 & 1 & 0\\
				\hline
				X & 9 & 9.5 & 9.9 & 10 & 11.5 & 12.7 & 13.4 & 13.7 & 14\\
				\hline
				Y & 1 & 0 & 1 & 0 & 1 & 1 & 1 & 1 & 0\\
				\hline
				X & 14.3 & 14.5 & 15.1 & 16.6 & 17.1 & 18.4 & 18.5 & 19.9 & 20\\
				\hline
				Y & 0 & 0 & 1 & 0 & 1 & 1 & 0 & 0 & 0\\
				\hline
			\end{tabular}\\[1em]
		\end{center}
	
		\paragraph{划分}
		\begin{center}
			\begin{tabular}{|c||c|c|c|c|c|c|c|c|c|c|}
				\hline
				X & 1 & 3 & 5 & 7 & 9 & 11 & 13 & 15 & 17 & 19\\
				\hline
				Y & 0.5 & 0.75 & 0 & 1 & 0.4 & 1 & 0.67 & 0.25 & 0.5 & 0.25\\
				\hline
			\end{tabular}\\[1em]
		\end{center}
		
		\paragraph{折线图}
		\begin{center}
			\begin{tikzpicture}
			\begin{axis}[
			title={X,Y},
			xlabel={},
			ylabel={},
			xtick={0,1,2,3,4,5,6,7,8,9,10,11,12,13,14,15,16,17,18,19,20},
			ytick={0,0.1,0.2,0.3,0.4,0.5,0.6,0.7,0.8,0.9,1},
			legend pos=north west,
			ymajorgrids=true,
			grid style=dashed,
			]
			\addplot[color=blue,mark=square]
			coordinates{(1,0.5)(3,0.75)(5,0)(7,1)(9,0.4)(11,1)(13,0.67)(15,0.25)(17,0.5)(19,0.25)};
			\legend{Y}
			\end{axis}
			\end{tikzpicture}
		\end{center}
		
		\paragraph{结果}\
		
		在整数精度下,程序运行结果:9
			
\end{document}