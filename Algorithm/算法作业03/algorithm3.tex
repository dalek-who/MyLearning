\documentclass{article} 
	\usepackage{fontspec,booktabs, xunicode, xltxtra,amsmath,amssymb,caption,colortbl,float,caption,algorithm,algpseudocode}
	\usepackage{graphicx}
	\usepackage{pgfplots}
	\usepackage{xeCJK}
	\setCJKmainfont{MSYH.ttc}
	\author{孔静-2014K8009929022}
	\title{Algorithm-homework2\\Greedy Algorithm}

	\definecolor{mypurple}{RGB}{112, 48, 160}
	\definecolor{myblue}{RGB}{12, 119, 195}
	\definecolor{myorange}{RGB}{197, 90, 17}
	\definecolor{myred}{RGB}{200, 31, 31}
	\definecolor{mygreen}{RGB}{84, 130, 53}
\begin{document}
	\maketitle 
	\tableofcontents
	\color{mypurple} \section{Greedy Algorithm}\color{black}
		\paragraph{a}\
		
			先经过一次最简单的判断,即$\sum\limits_{i = 1}^n {{d_i}}$是否是偶数。
			
			若是偶数:
			\[Opt(D,n) = \left\{ \begin{array}{l}
			1,{\rm{ }}n = 0\\
			0,{\rm{ }}{d_{\max }} \ge n\\
			Opt(D',n - 1),otherwise
			\end{array} \right.\]
			
			其中D是已经排序好的${d_1},{d_2}...{d_n}$递增序列。
			
			\[{d_i}' = \left\{ {\begin{array}{*{20}{l}}
				{0,i = n}\\
				{{d_i} - 1,n - {d_n} \le i < n}\\
				{{d_i},i < n - {d_n}}
				\end{array}} \right.\]
		
		\paragraph{b}\
		
			先判断是否已经能判断了,如果最大的度大于等于结点数,那么必然不能成图。然后,将度序列排序成递增序列,将度最大的节点${d_{\max }}$即${d_n}$删去,同时对剩余度最大的${d_n}$个减1。相当于从图上删去某节点以及与它相连的线。然后重复这个过程。
			\begin{algorithm}
				\caption{Problem1}
				\begin{algorithmic}
				\Procedure {Problem}{D,n}
				\If {$add(D) / 2 != 0$}
				\State Opt[n+1] = 0 
				\Else Opt[n+1] = 1 
				\EndIf
				\For {i = n to 1}
				\State quicksort(D,n)  \Comment 对前n个数,快速排序
				\If	{d(n) >= n || d(1) < 0 || Opt[i+1] == 0}
				\State Opt[i] = 0
				\Else
				\State Opt[i] = 1
				\EndIf
				\For {i = n - d(n) to n - 1}
				\State d(i) = d(i) - 1
				\EndFor
				\State n = n - 1
				\EndFor
				\State \textbf{return} $Opt[1]$\Comment 如果能构成图,应该输出1
				\EndProcedure
				\end{algorithmic}
			\end{algorithm}
		
		\paragraph{c}\
			
			下面证明为何删去最大度的节点,不影响是否成图。
			
			考虑度最大的节点n,如果它与剩下度最大的${d_n}$个点,有两种情况,一他们相连,那么删去正好。二他们不相连,n节点还与前$n-{d_n}$中有相连,下考虑情况2。A点属于剩下度最大的${d_n}$个点中,B属于前前$n-{d_n}$个点。
			
			\begin{tikzpicture}[
				rednode/.style={circle, draw=red, very thick, minimum size=5mm},
				orangenode/.style={circle, draw=orange, very thick, minimum size=5mm},
				greennode/.style={circle, draw=green, very thick, minimum size=5mm},
				]
				%Nodes
				\node[rednode] (main1) at (0,0) {${d_n}$};
				\node[rednode] (p0) at (-1,1) {A};
				\node[rednode] (p1) at (1,1) {B};
				%Lines
				\draw (main1)--(p1);
				%Nodes
				\node[orangenode] (main2) at (5,0) {${d_n}$};
				\node[orangenode] (p2) at (4,1) {A};
				\node[orangenode] (p3) at (6,1) {B};
				%Lines
				\draw (main2)--(p3);
				\draw (p2)--(p3);
			\end{tikzpicture}
			
			情况2根据B和A相连不相连又有两种情况:
			
			一(红色),从递增序列来看,A的度大于等于B,图中B已经少了一个度,所以在剩余n-3个点中,必然有一个结点C与A相连,不与B相连。将AC连线,B${d_n}$连线删去,改为A${d_n}$,BC,不影响度数。
			
			二(橙色),将${d_n}$与A相连,${d_n}$和A均与B不连,将B插入任意两个和B不相连的节点的连线中即可。首先${d_n}$已经与A不相连,B的度比${d_n}$小,那么至少存在一个与B不相连的结点,如果不存在第二个与B不相连的结点,说明B的度与${d_n}$相同,那么也与A同,将B和A结点位置互换即可。
			
			综上这种删去最大度的节点,并删去它所连的线的方法,对是否成图不影响。所以能利用该方法减少结点数,判断是否成图。
			
		\paragraph{d}\
		
			快排O(nlogn),删除节点O(n),遍历O(n)。
			
			综上算法复杂度是O(${n^2}logn$)。

	\color{myblue} \section{Greedy Algorithm} \color{black}
		\paragraph{a}\
		
			先根据${f_i}$从大到小排序。
			
			\[Opt(i,1) = \max \{ Opt(i - 1,0) + {s_i} + {f_i},Opt(i - 1,1)\} \]
		\paragraph{b}\
			\begin{algorithm}
				\caption{Greedy Algorithm}
				\begin{algorithmic}
					\Procedure {Greedy Algorithm}{S,F,n}
					\State quicksort(S,F)  \Comment 根据${f_i}$从大到小排序
					\For {i = 1 to n}
					\State Opt[i][1] = max(Opt[i-1][0] + s[i] + f[i], Opt[i-1][1])
					\State Opt[i][0] = Opt[i-1][0] + s[i]
					\EndFor
					\State \textbf{return} $Opt[n]$
					\EndProcedure
				\end{algorithmic}
			\end{algorithm}
	
		\paragraph{c}\
		
			\begin{tikzpicture}
				\draw[red] (0,0)--(4,0);
				\draw[orange] (4,0)--(10,0);
				\draw[red] (0,-1)--(3,-1);
				\draw[orange] (3,-1)--(5,-1);
			\end{tikzpicture}
		
			如图(红色是supercomputer运行时间,橙色是PC运行时间,显然应该是PC运行时间长的第一条线先运行)
			
			因为不考虑切换任务以及交替任务的时间,所以超级计算机运行的总时间是固定的,即$\sum\limits_{i = 1}^n {{s_i}}$,所以总共需要的时间,取决于普通计算机的运行结束时间,那当然是需要运行普通计算机最长时间的任务最先开始了。
			
		\paragraph{d}\
		
			快排O(nlogn),遍历O(n)。
		
			综上算法复杂度是O(nlogn)。
	\color{mygreen} \section{Greedy Algorithm} \color{black}
		\paragraph{a}\
		
		先根据岛屿的x轴坐标从小到大排序,并计算以每个岛屿i为圆心以d为半径的圆与x轴的交点,记为${left_i}$,和${right_i}$。
		
		\[Opt(i) = \left\{ \begin{array}{l}
		Opt(i - 1) + 1,lef{t_i} > {l_{Opt(i - 1)}}\\
		Opt(i - 1),lef{t_i} \le {l_{Opt(i - 1)}}
		\end{array} \right.\]
		
		其中${l_i}$记录的是雷达坐标,${l_1}$是${right_1}$,之后是Opt(i)雷达数量增加1,令${l_{Opt(i)}}$
	
		\paragraph{b}\
			\begin{algorithm}
				\caption{Greedy Algorithm}
				\begin{algorithmic}
					\Procedure {Greedy Algorithm}{X,Y,n}
					\State quicksort(X,Y,n)  \Comment 根据x坐标从小到大快速排序
					\State l[1] = right[1]
					\State Opt[1] = 1
					\For {i = 2 to n}
					\If {left[i] > l[Opt[i-1]]}
					\State Opt[i] = Opt[i-1] + 1
					\State l[Opt[i]] = right[i]
					\ElsIf {right[i] < l[Opt[i-1]]}
					\State Opt[i] = Opt[i-1]
					\State l[Opt[i]] = right[i]
					\Else
					\State Opt[i] = Opt[i-1]
					\EndIf
					\EndFor
					\State \textbf{return} $Opt[n]$
					\EndProcedure
				\end{algorithmic}
			\end{algorithm}
		\paragraph{c}\
			
			从x坐标最小的岛屿开始考虑,雷达必覆盖这个岛屿,并且尽可能多的去覆盖其他岛屿,所以雷达建立在这个岛屿对应的${right_1}$,之后一样,${left_i}$在这个雷达右边的,就表示该雷达覆盖不了,必须新建雷达;如果${left_i}$在雷达左边,分两种情况,一是${right_i}$在雷达右边,即该雷达能覆盖这个岛屿,二是${right_i}$在雷达左变,因为我们排序是按照岛屿横坐标排序的,此情况下,把该雷达移动到${right_i}$上即可以覆盖之前的岛屿也能覆盖该岛屿了。
		
		\paragraph{d}\
			
			快速排序O(nlogn), 遍历是O(n)。
			
			综上算法复杂度是O(nlogn)。
		
	\color{myorange} \section{Greedy Algorithm} \color{black}
		\paragraph{a}对A,B排序即可。
		
		\paragraph{b}排序就好,还要算法吗。。。ORZ。。。
	
		\paragraph{c}取对数,$\sum\limits_{i = 1}^n {{b_i} \times \ln ({a_i})}$,显然当${a_i}$和${b_i}$同为第k大的数时,该式最大。
		
		\paragraph{d}排序O(nlogn),所以算法复杂度是O(nlogn)。
			
\end{document}