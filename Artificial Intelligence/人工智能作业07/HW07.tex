\documentclass{article} 
	\usepackage{fontspec,booktabs, xunicode, xltxtra,amsmath,amssymb,caption,colortbl,float,caption,algorithm,algpseudocode}
	\usepackage{graphicx}
	\usepackage{pgfplots}
	\usepackage{xeCJK}
	\setCJKmainfont{MSYH.ttc}
	\author{孔静——2014K8009929022}
	\title{人工智能——模式挖掘的作业}

\begin{document}   
	\maketitle
	\tableofcontents
	\section{概\ 述}
		\subsection{问题}
		设计一个Transaction Database:
		
		1.画出它的lexicographic subset tree

		2.给定min\_sup, min\_occu的阈值,计算出在这些阈值下的frequent patterns, dominant patterns, maximal frequent patterns
		
		\subsection{要求}
		\paragraph{\ \ } 这个例子中,必须包含一个pattern,使得:它是maximal frequent pattern,但它不是dominant pattern
	\section{构\ 造}
		\subsection{故事背景}
		\paragraph{\ \ } 作为一个曾经的游戏玩家,经常购买游戏皮肤,现以玩家购买游戏皮肤为背景,创建数据。
		\subsection{商品种类}
		\paragraph{\ \ } 截止至我不玩英雄联盟,该游戏总共有655款皮肤。为简化作业复杂度,简化为五种商品,记为A,B,C,D,E。
		\subsection{购买记录}
		\paragraph{\ \ } 编造了一些数据如下表。
		\begin{center}
			\begin{tabular}{|c|c||c|c|}
				\hline
				玩家 & 商品 & 玩家 & 商品\\
				\hline
				1 & A B & 6 & A D\\
				\hline
				2 & A C D & 7 & E\\
				\hline
				3 & B E & 8 & A C E\\
				\hline
				4 & A D E & 9 & B D\\
				\hline
				5 & B C D E & 10 & A B C D E\\
				\hline
			\end{tabular}\\[1em]
		\end{center}
		\subsection{给定阈值} 
		\paragraph{\ \ } min\_sup\ =\ 0.3,min\_occu\ =\ 0.5
	\section{计\ 算}
		\subsection{画图}
		\begin{tikzpicture}
		\node[fill=gray] (o) at (-3,0) {O};
		\node[fill=lightgray] (a1) at (-7,-1) {A};
		\node[fill=lightgray] (a2) at (-2,-1) {B};
		\node[fill=lightgray] (a3) at (0.5,-1) {C};
		\node[fill=lightgray] (a4) at (2,-1) {D};
		\node[fill=lightgray] (a5) at (3,-1) {E};
		\node[fill=lightgray] (b1) at (-10,-2) {AB};
		\node[fill=lightgray] (b2) at (-8,-2) {AC};
		\node[fill=lightgray] (b3) at (-6,-2) {AD};
		\node[fill=lightgray] (b4) at (-4,-2) {AE};
		\node[fill=lightgray] (b5) at (-3,-2) {BC};
		\node[fill=lightgray] (b6) at (-2,-2) {BD};
		\node[fill=lightgray] (b7) at (-1,-2) {BE};
		\node[fill=lightgray] (b8) at (0,-2) {CD};
		\node[fill=lightgray] (b9) at (1,-2) {CE};
		\node[fill=lightgray] (b10) at (2,-2) {DE};
		\node[fill=lightgray] (c1) at (-11.5,-3) {ABC};
		\node[fill=lightgray] (c2) at (-10,-3) {ABD};
		\node[fill=lightgray] (c3) at (-8.5,-3) {ABE};
		\node[fill=lightgray] (c4) at (-7.25,-3) {ACD};
		\node[fill=lightgray] (c5) at (-6,-3) {ACE};
		\node[fill=lightgray] (c6) at (-4.75,-3) {ADE};
		\node[fill=lightgray] (c7) at (-3.5,-3) {BCD};
		\node[fill=lightgray] (c8) at (-2.25,-3) {BCE};
		\node[fill=lightgray] (c9) at (-1,-3) {BDE};
		\node[fill=lightgray] (c10) at (0.5,-3) {CDE};
		\node[fill=lightgray] (d1) at (-12.5,-4) {ABCD};
		\node[fill=lightgray] (d2) at (-10.5,-4) {ABCE};
		\node[fill=lightgray] (d3) at (-9,-4) {ABDE};
		\node[fill=lightgray] (d4) at (-7.25,-4) {ACDE};
		\node[fill=lightgray] (d5) at (-3.5,-4) {BCDE};
		\node[fill=lightgray] (e) at (-12.5,-5) {ABCDE};
		\draw[->] (o)--(a1);
		\draw[->] (o)--(a2);
		\draw[->] (o)--(a3);
		\draw[->] (o)--(a4);
		\draw[->] (o)--(a5);
		\draw[->] (a1)--(b1);
		\draw[->] (a1)--(b2);
		\draw[->] (a1)--(b3);
		\draw[->] (a1)--(b4);
		\draw[->] (a2)--(b5);
		\draw[->] (a2)--(b6);
		\draw[->] (a2)--(b7);
		\draw[->] (a3)--(b8);
		\draw[->] (a3)--(b9);
		\draw[->] (a4)--(b10);
		\draw[->] (b1)--(c1);
		\draw[->] (b1)--(c2);
		\draw[->] (b1)--(c3);
		\draw[->] (b2)--(c4);
		\draw[->] (b2)--(c5);
		\draw[->] (b3)--(c6);
		\draw[->] (b5)--(c7);
		\draw[->] (b5)--(c8);
		\draw[->] (b6)--(c9);
		\draw[->] (b8)--(c10);
		\draw[->] (c1)--(d1);
		\draw[->] (c1)--(d2);
		\draw[->] (c2)--(d3);
		\draw[->] (c4)--(d4);
		\draw[->] (c7)--(d5);
		\draw[->] (d1)--(e);
		\draw[color=red] (-9,-1.5) to[out=25,in=-180] (-6,-2.5);
		\draw[color=red] (-6,-2.5) to[out=0,in=-180] (-4,-1.5);
		\draw[color=red] (-4,-1.5) to[out=0,in=-5] (4,-1.5);
		\end{tikzpicture}
		\subsection{frequent patterns}
		\begin{center}
			\begin{tabular}{|c||c|c|c|c|c|}
				\hline
				1 & A & B & C & D & E\\
				\hline
				  & 0.6 & 0.5 & 0.4 & 0.6 & 0.6\\
				\hline
				2 & AB & AC & AD & AE & BC\\
				\hline
				  & 0.2 & 0.3 & 0.4 & 0.3 & 0.2\\
				\hline
				2 & BD & BE & CD & CE & DE\\
				\hline
	   			  & 0.3 & 0.3 & 0.3 & 0.3 & 0.3\\
				\hline
			\end{tabular}\\[1em]
		结果为:\\
		A,B,C,D,E\\
		AD
		\end{center}
		\subsection{dominant patterns}
				\begin{center}
			\begin{tabular}{|c||c|c|c|c|c|}
				\hline
				1 & A & B & C & D & E\\
				\hline
				  & $\frac{11}{30}$ & 0.39 & $\frac{67}{240}$ & $\frac{127}{360}$ & $\frac{157}{360}$\\
				\hline
				2 & AB & AC & AD & AE & BC\\
				\hline
				  & 0.7 & $\frac{26}{45}$ & $\frac{41}{60}$ & $\frac{26}{45}$ & 0.45\\
				\hline
				2 & BD & BE & CD & CE & DE\\
				\hline
				  & $\frac{19}{30}$ & $\frac{19}{30}$ & $\frac{47}{90}$ & $\frac{47}{90}$ & $\frac{47}{90}$\\
				\hline
				3 & ABC & ABD & ABE & ACD & ACE\\
				\hline
				  & 0.6 & 0.6 & 0.6 & 0.8 & 0.8\\
				\hline
				3 & ADE & BCD & BCE & BDE & CDE\\
				\hline
				  & 0.8 & 0.675 & 0.675 & 0.675 & 0.6\\
				\hline
				4 & ABCD & ABCE & ABDE & ACDE & BCDE\\
				\hline
				  & 0.8 & 0.8 & 0.8 & 0.8 & 0.9\\
				\hline
				5 & ABCDE & & & &\\
				\hline
				& 1 & & & &\\
				\hline
			\end{tabular}\\[1em]
			结果为:\\
			AB,AC,AD,AE,
			BD,BE,CD,CE,DE\\
			ABC,ABD,ABE,ACD,ACE,
			ADE,BCD,BCE,BDE,CDE\\
			ABCD,ABCE,ABDE,ACDE,BCDE\\
			ABCDE
		\end{center}
		\subsection{maximal frequent patterns}
		\begin{center}
			结果为:B,C,E,AD
		\end{center}
		\subsection{others}
		\begin{center}
			显然,B,C,E是maximal frequent patterns,而不是dominant patterns
		\end{center}
	\end{document}
\end{document}